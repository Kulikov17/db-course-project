% Также можно использовать \Referat, как в оригинале
\Referat

\hfill

Курсовой проект представляет собой микросервис уни­версального мессенджера.

Ключевые слова: микросервис, мессенджер, универсальный, PostreSQL, ASP.NET Core, Entity Framework Core, REST. 

Данный микросервис реализуется на языке программирования C$\#$  с­ использованием платформы ASP.NET Core и технологии Entity Framework Core. Для взаимодействия между клиентом и сервером используется архитектурный стиль REST.

Полученное в результате работы ПО может быть использовано как микросервис для создании полноценного корпоративного мессенджера или любого другого приложения, где необходим функционал мгновенного обмена сообщениями. 

Отчёт содержит \pageref{LastPage}\,~страниц%
    \ifnum \totfig >0
    , \totfig~рисунка%
    \fi
    \ifnum \tottab >0
    , \tottab~таблицы%
    \fi
    %
    \ifnum \totbib >0
    , \totbib~источников%
    \fi
    %
    \ifnum \totapp >0
    , \totapp~прил.%
    \else
    .%
    \fi

%%% Local Variables: 
%%% mode: latex
%%% TeX-master: "rpz"
%%% End: 
